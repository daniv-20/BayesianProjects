% Options for packages loaded elsewhere
\PassOptionsToPackage{unicode}{hyperref}
\PassOptionsToPackage{hyphens}{url}
%
\documentclass[
]{article}
\usepackage{amsmath,amssymb}
\usepackage{iftex}
\ifPDFTeX
  \usepackage[T1]{fontenc}
  \usepackage[utf8]{inputenc}
  \usepackage{textcomp} % provide euro and other symbols
\else % if luatex or xetex
  \usepackage{unicode-math} % this also loads fontspec
  \defaultfontfeatures{Scale=MatchLowercase}
  \defaultfontfeatures[\rmfamily]{Ligatures=TeX,Scale=1}
\fi
\usepackage{lmodern}
\ifPDFTeX\else
  % xetex/luatex font selection
\fi
% Use upquote if available, for straight quotes in verbatim environments
\IfFileExists{upquote.sty}{\usepackage{upquote}}{}
\IfFileExists{microtype.sty}{% use microtype if available
  \usepackage[]{microtype}
  \UseMicrotypeSet[protrusion]{basicmath} % disable protrusion for tt fonts
}{}
\makeatletter
\@ifundefined{KOMAClassName}{% if non-KOMA class
  \IfFileExists{parskip.sty}{%
    \usepackage{parskip}
  }{% else
    \setlength{\parindent}{0pt}
    \setlength{\parskip}{6pt plus 2pt minus 1pt}}
}{% if KOMA class
  \KOMAoptions{parskip=half}}
\makeatother
\usepackage{xcolor}
\usepackage[margin=1in]{geometry}
\usepackage{color}
\usepackage{fancyvrb}
\newcommand{\VerbBar}{|}
\newcommand{\VERB}{\Verb[commandchars=\\\{\}]}
\DefineVerbatimEnvironment{Highlighting}{Verbatim}{commandchars=\\\{\}}
% Add ',fontsize=\small' for more characters per line
\usepackage{framed}
\definecolor{shadecolor}{RGB}{248,248,248}
\newenvironment{Shaded}{\begin{snugshade}}{\end{snugshade}}
\newcommand{\AlertTok}[1]{\textcolor[rgb]{0.94,0.16,0.16}{#1}}
\newcommand{\AnnotationTok}[1]{\textcolor[rgb]{0.56,0.35,0.01}{\textbf{\textit{#1}}}}
\newcommand{\AttributeTok}[1]{\textcolor[rgb]{0.13,0.29,0.53}{#1}}
\newcommand{\BaseNTok}[1]{\textcolor[rgb]{0.00,0.00,0.81}{#1}}
\newcommand{\BuiltInTok}[1]{#1}
\newcommand{\CharTok}[1]{\textcolor[rgb]{0.31,0.60,0.02}{#1}}
\newcommand{\CommentTok}[1]{\textcolor[rgb]{0.56,0.35,0.01}{\textit{#1}}}
\newcommand{\CommentVarTok}[1]{\textcolor[rgb]{0.56,0.35,0.01}{\textbf{\textit{#1}}}}
\newcommand{\ConstantTok}[1]{\textcolor[rgb]{0.56,0.35,0.01}{#1}}
\newcommand{\ControlFlowTok}[1]{\textcolor[rgb]{0.13,0.29,0.53}{\textbf{#1}}}
\newcommand{\DataTypeTok}[1]{\textcolor[rgb]{0.13,0.29,0.53}{#1}}
\newcommand{\DecValTok}[1]{\textcolor[rgb]{0.00,0.00,0.81}{#1}}
\newcommand{\DocumentationTok}[1]{\textcolor[rgb]{0.56,0.35,0.01}{\textbf{\textit{#1}}}}
\newcommand{\ErrorTok}[1]{\textcolor[rgb]{0.64,0.00,0.00}{\textbf{#1}}}
\newcommand{\ExtensionTok}[1]{#1}
\newcommand{\FloatTok}[1]{\textcolor[rgb]{0.00,0.00,0.81}{#1}}
\newcommand{\FunctionTok}[1]{\textcolor[rgb]{0.13,0.29,0.53}{\textbf{#1}}}
\newcommand{\ImportTok}[1]{#1}
\newcommand{\InformationTok}[1]{\textcolor[rgb]{0.56,0.35,0.01}{\textbf{\textit{#1}}}}
\newcommand{\KeywordTok}[1]{\textcolor[rgb]{0.13,0.29,0.53}{\textbf{#1}}}
\newcommand{\NormalTok}[1]{#1}
\newcommand{\OperatorTok}[1]{\textcolor[rgb]{0.81,0.36,0.00}{\textbf{#1}}}
\newcommand{\OtherTok}[1]{\textcolor[rgb]{0.56,0.35,0.01}{#1}}
\newcommand{\PreprocessorTok}[1]{\textcolor[rgb]{0.56,0.35,0.01}{\textit{#1}}}
\newcommand{\RegionMarkerTok}[1]{#1}
\newcommand{\SpecialCharTok}[1]{\textcolor[rgb]{0.81,0.36,0.00}{\textbf{#1}}}
\newcommand{\SpecialStringTok}[1]{\textcolor[rgb]{0.31,0.60,0.02}{#1}}
\newcommand{\StringTok}[1]{\textcolor[rgb]{0.31,0.60,0.02}{#1}}
\newcommand{\VariableTok}[1]{\textcolor[rgb]{0.00,0.00,0.00}{#1}}
\newcommand{\VerbatimStringTok}[1]{\textcolor[rgb]{0.31,0.60,0.02}{#1}}
\newcommand{\WarningTok}[1]{\textcolor[rgb]{0.56,0.35,0.01}{\textbf{\textit{#1}}}}
\usepackage{graphicx}
\makeatletter
\def\maxwidth{\ifdim\Gin@nat@width>\linewidth\linewidth\else\Gin@nat@width\fi}
\def\maxheight{\ifdim\Gin@nat@height>\textheight\textheight\else\Gin@nat@height\fi}
\makeatother
% Scale images if necessary, so that they will not overflow the page
% margins by default, and it is still possible to overwrite the defaults
% using explicit options in \includegraphics[width, height, ...]{}
\setkeys{Gin}{width=\maxwidth,height=\maxheight,keepaspectratio}
% Set default figure placement to htbp
\makeatletter
\def\fps@figure{htbp}
\makeatother
\setlength{\emergencystretch}{3em} % prevent overfull lines
\providecommand{\tightlist}{%
  \setlength{\itemsep}{0pt}\setlength{\parskip}{0pt}}
\setcounter{secnumdepth}{5}
\usepackage{multirow}
\usepackage{multicol}
\usepackage{colortbl}
\usepackage{hhline}
\newlength\Oldarrayrulewidth
\newlength\Oldtabcolsep
\usepackage{longtable}
\usepackage{array}
\usepackage{hyperref}
\usepackage{float}
\usepackage{wrapfig}
\ifLuaTeX
  \usepackage{selnolig}  % disable illegal ligatures
\fi
\IfFileExists{bookmark.sty}{\usepackage{bookmark}}{\usepackage{hyperref}}
\IfFileExists{xurl.sty}{\usepackage{xurl}}{} % add URL line breaks if available
\urlstyle{same}
\hypersetup{
  pdftitle={Project 2},
  pdfauthor={Sam Albertson, Suz Angermeier, \& Dani Vaithilingam},
  hidelinks,
  pdfcreator={LaTeX via pandoc}}

\title{Project 2}
\author{Sam Albertson, Suz Angermeier, \& Dani Vaithilingam}
\date{3/10/2023}

\begin{document}
\maketitle

{
\setcounter{tocdepth}{3}
\tableofcontents
}
\hypertarget{safety-monitoring-rule}{%
\section{Safety Monitoring Rule}\label{safety-monitoring-rule}}

\hypertarget{objective-of-the-safety-monitoring-rule}{%
\subsection{Objective of the Safety Monitoring
Rule}\label{objective-of-the-safety-monitoring-rule}}

The goal of the safety monitoring rule is to set a trigger for
determining when the rate of severe adverse events is too high. A
maximum acceptable rate is pre-specified, and if we are confident that
the probability exceeds this level, the study is discontinued.

\hypertarget{parameter-to-estimate-and-hypothesis}{%
\subsection{Parameter to Estimate and
Hypothesis}\label{parameter-to-estimate-and-hypothesis}}

The parameter we try to estimate for the safety monitoring rule is the
probability \(\theta\) that a patient will have an adverse event of
Grade 3 (severe) or higher. We define our maximum clinically acceptable
risk level as \(\theta_{mcid}\) = 0.05; our safety monitoring rule is
triggered if there is a greater than 80\% chance that
\(\theta > \theta_{mcid}\).

\hypertarget{statistical-model}{%
\subsection{Statistical Model}\label{statistical-model}}

The beta-binomial model is a functional family of a prior (beta
distribution) and likelihood (binomial distribution) function. These are
multiplied together at a range of values of \(\theta\) to generate our
posterior distribution, from which we can estimate the probability that
\(\theta > \theta_{mcid}\)

\hypertarget{prior}{%
\subsubsection{Prior}\label{prior}}

The prior distribution represents our knowledge of the adverse event
rate before conducting the study. We use a \(Beta(1,1)\) prior, which
contains no information about the parameter, to represent our lack of
information about the event rate.

\hypertarget{likelihood}{%
\subsubsection{Likelihood}\label{likelihood}}

The likelihood function represents the chance of observing the number of
events seen in the trial, given a proposed risk level. We define this
mathematically using a \(Binomial(N_{events}, N_{total}, \theta)\)
distribution. This provides an empirical estimate of the event rate
parameter.

\hypertarget{posterior-distribution}{%
\subsubsection{Posterior Distribution}\label{posterior-distribution}}

We define the posterior distribution as the product of the prior and the
likelihood of our data:

\(P[Y|X] \sim P[Y] * L[Y | X]\)

\hypertarget{critical-boundary}{%
\subsubsection{Critical Boundary}\label{critical-boundary}}

\begin{Shaded}
\begin{Highlighting}[]
\NormalTok{f\_post }\OtherTok{\textless{}{-}} \ControlFlowTok{function}\NormalTok{(x)\{}
\NormalTok{  pct }\OtherTok{\textless{}{-}} \FunctionTok{signif}\NormalTok{(}\DecValTok{1} \SpecialCharTok{{-}} \FunctionTok{pbeta}\NormalTok{(}\FloatTok{0.05}\NormalTok{, x }\SpecialCharTok{+} \DecValTok{1}\NormalTok{, }\DecValTok{50} \SpecialCharTok{{-}}\NormalTok{ x }\SpecialCharTok{+} \DecValTok{1}\NormalTok{), }\DecValTok{3}\NormalTok{)}
  \FunctionTok{return}\NormalTok{(}\FunctionTok{paste0}\NormalTok{(pct, }\StringTok{"\%"}\NormalTok{))}
\NormalTok{\}}
\NormalTok{df\_post }\OtherTok{\textless{}{-}} \FunctionTok{data.frame}\NormalTok{(}\AttributeTok{Events =} \DecValTok{0}\SpecialCharTok{:}\DecValTok{50}\NormalTok{) }\SpecialCharTok{\%\textgreater{}\%}
  \FunctionTok{mutate}\NormalTok{(}\StringTok{\textasciigrave{}}\AttributeTok{Posterior Probability}\StringTok{\textasciigrave{}} \OtherTok{=} \FunctionTok{f\_post}\NormalTok{(Events))}
\NormalTok{df\_post[}\DecValTok{1}\SpecialCharTok{:}\DecValTok{6}\NormalTok{,]}
\end{Highlighting}
\end{Shaded}

\begin{verbatim}
##   Events Posterior Probability
## 1      0               0.0731%
## 2      1                0.269%
## 3      2                0.527%
## 4      3                0.749%
## 5      4                 0.89%
## 6      5                0.959%
\end{verbatim}

In the above table we see the probability that the risk level is greater
than the MCID. Our safety monitoring rule is triggered after this
probability is greater than 80\%. We can see that we are below 80\% when
there is 0 to 3 events, once 4 event s are present in the data our
probability is greater than 80\%. This only increases as the number of
events increase. This tells us that the critical boundary is 4 events
for the study.

\newpage

\hypertarget{operating-characteristics-of-the-efficacy-analysis}{%
\section{Operating Characteristics of the Efficacy
Analysis}\label{operating-characteristics-of-the-efficacy-analysis}}

\hypertarget{objective}{%
\subsection{Objective}\label{objective}}

\emph{Describe in your own words what the objective of the efficacy
analysis is.}

The objective of the efficacy analysis is to determine if the continuous
abstinence from smoking across the last 4 weeks of the study is
sufficient to conclude that the QUITNOW device has enough benefit to be
worth further study.

\hypertarget{parameter-to-estimate-and-hypothesis-1}{%
\subsection{Parameter to Estimate and
Hypothesis}\label{parameter-to-estimate-and-hypothesis-1}}

\emph{Describe the parameter to be estimated in the efficacy analysis,
the minimum clinically relevant value for the parameter, and the
hypothesis to be tested. What are the success criteria for the efficacy
analysis?}

The parameter to be estimated in the efficacy analysis is the proportion
of participants who quit smoking cigarettes at week 27. The minimal
clinically relevant value is 12.4\% (\(\theta\) \textgreater{} .124),
and the hypothesis is that the abstinence rate will be greater than
12.4\%. The quantitative success criterion is that P(\(\theta\)
\textgreater{} .124) \textgreater{} 0.8, or that there is an 80\%
probability of the true quit rate being greater than 12.4\%.

\hypertarget{statistical-model-1}{%
\subsection{Statistical Model}\label{statistical-model-1}}

\emph{Show the statistical model for the analysis, including prior,
likelihood and posterior distribution. State these mathematically and in
plain English as they relate to the objectives of the study. Note, there
are 3 priors discussed in the protocol: one used for the primary
analysis and two used for sensitivity analyses, the ``optimistic'' and
``pessimisic'' priors. Discuss why these priors are labeled as
optimistic or pessimistic and what impact you think they might have on
the posterior distribution.}

\textbf{Priors} Primary analysis: \(Beta(1,1)\) Optimistic:
\(Beta(6,9)\) Pessimistic: \(Beta(4,28)\)

\textbf{Likelihood Function}
\(f(y|\pi) = P(Y=y|\pi)=\binom{50}{y}\pi^y(1-\pi)^{50-y}\)

\textbf{Posteriors}

Primary analysis: \(Beta(1,1)*f(y|\pi)\propto Beta(1+y, 1+50-y)\)
Optimistic: \(Beta(6,9)*f(y|\pi)\propto Beta(6-y, 9+50-y)\) Pessimistic:
\(Beta(4,28)*f(y|\pi)\propto Beta(4-y, 28+50-y)\)

The reason for having optimistic and pessimistic priors is that we do
not actually know if the QUITNOW device will perform more similarly to
e-cigarettes in previous trials (optimistic) or placebo in previous
trials (pessimistic).

Because we do not have prior data on the QUITNOW device, we are using a
non-informative beta(1,1) prior for our main analysis, however the prior
does have some weight and therefore has the potential to give us a
different posterior distribution for the same data than if we assume
that the QUITNOW device will perform more similarly to either the
e-cigarettes or placebo.

We will be performing a sensitivity analysis using the optimistic and
pessimistic priors to determine if the three resulting posteriors allow
us to draw the same final conclusion (for example, that the QUITNOW
device is effective). If so, we can be more confident in our results. If
the three results do not agree, we may be less sure of the validity of
our results.

\hypertarget{definition-of-power-and-type-i-error-in-the-context-of-the-efficacy-analysis}{%
\subsection{Definition of Power and Type I Error in the Context of the
Efficacy
Analysis}\label{definition-of-power-and-type-i-error-in-the-context-of-the-efficacy-analysis}}

\emph{Any stochastic rule for success or failure could lead to an
incorrect conclusion. How are the possibilities for drawing an incorrect
conclusion represented in the Bayesian framework for this trial? Explain
this plan English.}

The type I error rate and power in bayesian analysis are determined
using repeated simulations of
\(X = \frac{1}{N} \sum_{i=1}^N I (Pr\{\theta> 0.124\} > 0.8)\) different
plausible \(\theta\) values for the QUITNOW device. When \(\theta\)
\textgreater{} 0.124, X is the statistical power, or the proportion of
simulations wherein the success criterion was met correctly and when
\(\theta\) \textless{} 0.123, X is the type I error rate, or the
proportion of simulations where the success criterion was met falsly.

In plain english: the power, or probability of correctly determining
that the success criterion was met, and type I error rate, the
probability of incorrectly determining that the success criterion was
met, in bayesian analysis are determined using repeated simulations of
the success criterion with each simulation taking a value of 1 when the
success criteria is met and 0 when it is not met. In these simulations,
values of \(\theta\) both above and below the clinical threshold of
0.124 are used, with each having different meanings. When \(\theta\)
\textgreater{} 0.124, we are determining the probability of concluding
success when the success criterion is correctly met, and when \(\theta\)
\textless{} 0.124, we are determining the probability of concluding
success when the success criterion is not correctly met.

\hypertarget{design-of-your-program-to-evaluate-the-operating-characteristics-of-the-efficacy-analysis}{%
\subsection{Design of Your Program to Evaluate the Operating
Characteristics of the Efficacy
Analysis}\label{design-of-your-program-to-evaluate-the-operating-characteristics-of-the-efficacy-analysis}}

For the program, we are using N = 10,000 simulations, with each
simulated study consisting of 50 participants, like the actual study
will.

\hypertarget{inputs}{%
\subsubsection{Inputs}\label{inputs}}

Describe the input that goes into the program.

The inputs into the program are as follows:

\begin{verbatim}
## Warning: fonts used in `flextable` are ignored because the `pdflatex` engine is
## used and not `xelatex` or `lualatex`. You can avoid this warning by using the
## `set_flextable_defaults(fonts_ignore=TRUE)` command or use a compatible engine
## by defining `latex_engine: xelatex` in the YAML header of the R Markdown
## document.
\end{verbatim}

\global\setlength{\Oldarrayrulewidth}{\arrayrulewidth}

\global\setlength{\Oldtabcolsep}{\tabcolsep}

\setlength{\tabcolsep}{2pt}

\renewcommand*{\arraystretch}{1.5}



\providecommand{\ascline}[3]{\noalign{\global\arrayrulewidth #1}\arrayrulecolor[HTML]{#2}\cline{#3}}

\begin{longtable}[c]{|p{0.75in}|p{0.75in}|p{0.75in}}



\ascline{1.5pt}{666666}{1-3}

\multicolumn{1}{>{\raggedright}m{\dimexpr 0.75in+0\tabcolsep}}{\textcolor[HTML]{000000}{\fontsize{11}{11}\selectfont{Input}}} & \multicolumn{1}{>{\raggedright}m{\dimexpr 0.75in+0\tabcolsep}}{\textcolor[HTML]{000000}{\fontsize{11}{11}\selectfont{Description}}} & \multicolumn{1}{>{\raggedright}m{\dimexpr 0.75in+0\tabcolsep}}{\textcolor[HTML]{000000}{\fontsize{11}{11}\selectfont{Value}}} \\

\ascline{1.5pt}{666666}{1-3}\endfirsthead 

\ascline{1.5pt}{666666}{1-3}

\multicolumn{1}{>{\raggedright}m{\dimexpr 0.75in+0\tabcolsep}}{\textcolor[HTML]{000000}{\fontsize{11}{11}\selectfont{Input}}} & \multicolumn{1}{>{\raggedright}m{\dimexpr 0.75in+0\tabcolsep}}{\textcolor[HTML]{000000}{\fontsize{11}{11}\selectfont{Description}}} & \multicolumn{1}{>{\raggedright}m{\dimexpr 0.75in+0\tabcolsep}}{\textcolor[HTML]{000000}{\fontsize{11}{11}\selectfont{Value}}} \\

\ascline{1.5pt}{666666}{1-3}\endhead



\multicolumn{1}{>{\raggedright}m{\dimexpr 0.75in+0\tabcolsep}}{\textcolor[HTML]{000000}{\fontsize{11}{11}\selectfont{N}}} & \multicolumn{1}{>{\raggedright}m{\dimexpr 0.75in+0\tabcolsep}}{\textcolor[HTML]{000000}{\fontsize{11}{11}\selectfont{Number\ of\ Simulations}}} & \multicolumn{1}{>{\raggedright}m{\dimexpr 0.75in+0\tabcolsep}}{\textcolor[HTML]{000000}{\fontsize{11}{11}\selectfont{10,000}}} \\





\multicolumn{1}{>{\raggedright}m{\dimexpr 0.75in+0\tabcolsep}}{\textcolor[HTML]{000000}{\fontsize{11}{11}\selectfont{SampleSize}}} & \multicolumn{1}{>{\raggedright}m{\dimexpr 0.75in+0\tabcolsep}}{\textcolor[HTML]{000000}{\fontsize{11}{11}\selectfont{Number\ of\ participants\ in\ simulated\ study}}} & \multicolumn{1}{>{\raggedright}m{\dimexpr 0.75in+0\tabcolsep}}{\textcolor[HTML]{000000}{\fontsize{11}{11}\selectfont{50}}} \\





\multicolumn{1}{>{\raggedright}m{\dimexpr 0.75in+0\tabcolsep}}{\textcolor[HTML]{000000}{\fontsize{11}{11}\selectfont{Theta.test\ (vector)}}} & \multicolumn{1}{>{\raggedright}m{\dimexpr 0.75in+0\tabcolsep}}{\textcolor[HTML]{000000}{\fontsize{11}{11}\selectfont{A\ vector\ containign\ a\ range\ of\ possible\ θ\ values.}}} & \multicolumn{1}{>{\raggedright}m{\dimexpr 0.75in+0\tabcolsep}}{\textcolor[HTML]{000000}{\fontsize{11}{11}\selectfont{Varies:\ 0<\ θ\ <1}}} \\





\multicolumn{1}{>{\raggedright}m{\dimexpr 0.75in+0\tabcolsep}}{\textcolor[HTML]{000000}{\fontsize{11}{11}\selectfont{Alpha.prior}}} & \multicolumn{1}{>{\raggedright}m{\dimexpr 0.75in+0\tabcolsep}}{\textcolor[HTML]{000000}{\fontsize{11}{11}\selectfont{A\ vector\ containing\ the\ alpha\ values\ of\ the\ priors}}} & \multicolumn{1}{>{\raggedright}m{\dimexpr 0.75in+0\tabcolsep}}{\textcolor[HTML]{000000}{\fontsize{11}{11}\selectfont{1,\ 6,\ 4}}} \\





\multicolumn{1}{>{\raggedright}m{\dimexpr 0.75in+0\tabcolsep}}{\textcolor[HTML]{000000}{\fontsize{11}{11}\selectfont{Beta.prior}}} & \multicolumn{1}{>{\raggedright}m{\dimexpr 0.75in+0\tabcolsep}}{\textcolor[HTML]{000000}{\fontsize{11}{11}\selectfont{A\ vector\ containing\ the\ beta\ values\ of\ the\ priors}}} & \multicolumn{1}{>{\raggedright}m{\dimexpr 0.75in+0\tabcolsep}}{\textcolor[HTML]{000000}{\fontsize{11}{11}\selectfont{1,\ 9,\ 28}}} \\





\multicolumn{1}{>{\raggedright}m{\dimexpr 0.75in+0\tabcolsep}}{\textcolor[HTML]{000000}{\fontsize{11}{11}\selectfont{Theta.success}}} & \multicolumn{1}{>{\raggedright}m{\dimexpr 0.75in+0\tabcolsep}}{\textcolor[HTML]{000000}{\fontsize{11}{11}\selectfont{The\ success\ criterion}}} & \multicolumn{1}{>{\raggedright}m{\dimexpr 0.75in+0\tabcolsep}}{\textcolor[HTML]{000000}{\fontsize{11}{11}\selectfont{0.124}}} \\

\ascline{1.5pt}{666666}{1-3}



\end{longtable}



\arrayrulecolor[HTML]{000000}

\global\setlength{\arrayrulewidth}{\Oldarrayrulewidth}

\global\setlength{\tabcolsep}{\Oldtabcolsep}

\renewcommand*{\arraystretch}{1}

\hypertarget{outputs}{%
\subsubsection{Outputs}\label{outputs}}

Describe what the program creates as a final result.

The program will output a table, with columns as described in the table
below.

\begin{verbatim}
## Warning: fonts used in `flextable` are ignored because the `pdflatex` engine is
## used and not `xelatex` or `lualatex`. You can avoid this warning by using the
## `set_flextable_defaults(fonts_ignore=TRUE)` command or use a compatible engine
## by defining `latex_engine: xelatex` in the YAML header of the R Markdown
## document.
\end{verbatim}

\global\setlength{\Oldarrayrulewidth}{\arrayrulewidth}

\global\setlength{\Oldtabcolsep}{\tabcolsep}

\setlength{\tabcolsep}{2pt}

\renewcommand*{\arraystretch}{1.5}



\providecommand{\ascline}[3]{\noalign{\global\arrayrulewidth #1}\arrayrulecolor[HTML]{#2}\cline{#3}}

\begin{longtable}[c]{|p{0.75in}|p{0.75in}}



\ascline{1.5pt}{666666}{1-2}

\multicolumn{1}{>{\raggedright}m{\dimexpr 0.75in+0\tabcolsep}}{\textcolor[HTML]{000000}{\fontsize{11}{11}\selectfont{Output}}} & \multicolumn{1}{>{\raggedright}m{\dimexpr 0.75in+0\tabcolsep}}{\textcolor[HTML]{000000}{\fontsize{11}{11}\selectfont{Description}}} \\

\ascline{1.5pt}{666666}{1-2}\endfirsthead 

\ascline{1.5pt}{666666}{1-2}

\multicolumn{1}{>{\raggedright}m{\dimexpr 0.75in+0\tabcolsep}}{\textcolor[HTML]{000000}{\fontsize{11}{11}\selectfont{Output}}} & \multicolumn{1}{>{\raggedright}m{\dimexpr 0.75in+0\tabcolsep}}{\textcolor[HTML]{000000}{\fontsize{11}{11}\selectfont{Description}}} \\

\ascline{1.5pt}{666666}{1-2}\endhead



\multicolumn{1}{>{\raggedright}m{\dimexpr 0.75in+0\tabcolsep}}{\textcolor[HTML]{000000}{\fontsize{11}{11}\selectfont{Theta.test}}} & \multicolumn{1}{>{\raggedright}m{\dimexpr 0.75in+0\tabcolsep}}{\textcolor[HTML]{000000}{\fontsize{11}{11}\selectfont{The\ value\ of\ theta\ being\ simulated}}} \\





\multicolumn{1}{>{\raggedright}m{\dimexpr 0.75in+0\tabcolsep}}{\textcolor[HTML]{000000}{\fontsize{11}{11}\selectfont{Result}}} & \multicolumn{1}{>{\raggedright}m{\dimexpr 0.75in+0\tabcolsep}}{\textcolor[HTML]{000000}{\fontsize{11}{11}\selectfont{Whether\ the\ type\ of\ result\ displayed\ is\ power\ or\ type\ I\ error}}} \\





\multicolumn{1}{>{\raggedright}m{\dimexpr 0.75in+0\tabcolsep}}{\textcolor[HTML]{000000}{\fontsize{11}{11}\selectfont{Value}}} & \multicolumn{1}{>{\raggedright}m{\dimexpr 0.75in+0\tabcolsep}}{\textcolor[HTML]{000000}{\fontsize{11}{11}\selectfont{The\ value\ of\ the\ simulated\ power/type\ I\ error}}} \\

\ascline{1.5pt}{666666}{1-2}



\end{longtable}



\arrayrulecolor[HTML]{000000}

\global\setlength{\arrayrulewidth}{\Oldarrayrulewidth}

\global\setlength{\tabcolsep}{\Oldtabcolsep}

\renewcommand*{\arraystretch}{1}

\hypertarget{algorithm}{%
\subsubsection{Algorithm}\label{algorithm}}

Describe how the program operates to produce output from the input. You
can use words or draw a diagram, or both.

The program works as follows:

For each value of theta in theta test: 1. Use rbinom() with N, Sample
Size and theta to get a vector of N simulated Y values where Y is the
number of simulated successes. 2. Use get\_beta\_post() (a self-made
function included in bayesSim.R) to get the posterior alpha and beta
using each of the primary, optimistic and pessimistic priors and the
simulated Y values 3. Calculate the probability of (\(\theta\)
\textgreater{} 0.124) for each posterior 4. Using those values, if the
probability is \textgreater{} 0.8, store 1 (success) for that trial and
if not store 0 for that trial 5. Calculate the proportion of successes
for the given value of theta

Return table containing theta values and proportion of successes for
each value of theta.

After all simulations have run,
\(X = \frac{1}{N} \sum_{i=1}^N I (Pr\{\theta> 0.124\} > 0.8)\) is
calculated for each value of\(\theta\), with the value of the indicator
function being the value in the success column. For each theta, the
value of X is stored in a data frame along with the value of theta and
whether the value of X is a power calculation or a type I error
calculation.

\hypertarget{program-code}{%
\subsection{Program Code}\label{program-code}}

\emph{Show your program code here.}

\begin{Shaded}
\begin{Highlighting}[]
\NormalTok{qn.sim }\OtherTok{=} \ControlFlowTok{function}\NormalTok{(}\AttributeTok{N =} \DecValTok{10000}\NormalTok{, }\AttributeTok{SampleSize =} \DecValTok{50}\NormalTok{, }\AttributeTok{Theta.test =} \FunctionTok{seq}\NormalTok{(}\FloatTok{0.01}\NormalTok{, }\FloatTok{0.99}\NormalTok{, }\FloatTok{0.01}\NormalTok{), }\AttributeTok{Alpha.prior =} \FunctionTok{c}\NormalTok{(}\DecValTok{1}\NormalTok{, }\DecValTok{6}\NormalTok{, }\DecValTok{4}\NormalTok{), }\AttributeTok{Beta.prior =} \FunctionTok{c}\NormalTok{(}\DecValTok{1}\NormalTok{, }\DecValTok{9}\NormalTok{, }\DecValTok{28}\NormalTok{), }\AttributeTok{Theta.success =} \FloatTok{0.124}\NormalTok{)\{}
\NormalTok{  simdat }\OtherTok{=} \FunctionTok{data.frame}\NormalTok{(}\AttributeTok{y =} \FunctionTok{rep}\NormalTok{(}\ConstantTok{NA}\NormalTok{, N))}
\NormalTok{  result }\OtherTok{=} \FunctionTok{data.frame}\NormalTok{(}\AttributeTok{theta =}\NormalTok{ Theta.test)}
\NormalTok{  i }\OtherTok{=} \DecValTok{1}
  \ControlFlowTok{for}\NormalTok{(theta }\ControlFlowTok{in}\NormalTok{ Theta.test)\{}
\NormalTok{    simdat}\SpecialCharTok{$}\NormalTok{y }\OtherTok{=} \FunctionTok{rbinom}\NormalTok{(N, SampleSize, theta)}
\NormalTok{    post.prim}\OtherTok{=} \FunctionTok{get\_beta\_post}\NormalTok{(Alpha.prior[}\DecValTok{1}\NormalTok{], Beta.prior[}\DecValTok{1}\NormalTok{], simdat}\SpecialCharTok{$}\NormalTok{y, SampleSize)}
\NormalTok{    post.opt }\OtherTok{=} \FunctionTok{get\_beta\_post}\NormalTok{(Alpha.prior[}\DecValTok{2}\NormalTok{], Beta.prior[}\DecValTok{2}\NormalTok{], simdat}\SpecialCharTok{$}\NormalTok{y, SampleSize)}
\NormalTok{    post.pes }\OtherTok{=} \FunctionTok{get\_beta\_post}\NormalTok{(Alpha.prior[}\DecValTok{3}\NormalTok{], Beta.prior[}\DecValTok{3}\NormalTok{], simdat}\SpecialCharTok{$}\NormalTok{y, SampleSize)}
\NormalTok{    simdat}\SpecialCharTok{$}\NormalTok{criterion.prim }\OtherTok{=} \FunctionTok{pbeta}\NormalTok{(}\FloatTok{0.124}\NormalTok{, post.prim}\SpecialCharTok{$}\NormalTok{alpha.post, post.prim}\SpecialCharTok{$}\NormalTok{beta.post, }\AttributeTok{lower.tail =} \ConstantTok{FALSE}\NormalTok{)}
\NormalTok{    simdat}\SpecialCharTok{$}\NormalTok{criterion.opt }\OtherTok{=} \FunctionTok{pbeta}\NormalTok{(}\FloatTok{0.124}\NormalTok{, post.opt}\SpecialCharTok{$}\NormalTok{alpha.post, post.opt}\SpecialCharTok{$}\NormalTok{beta.post, }\AttributeTok{lower.tail =} \ConstantTok{FALSE}\NormalTok{)}
\NormalTok{    simdat}\SpecialCharTok{$}\NormalTok{criterion.pes }\OtherTok{=} \FunctionTok{pbeta}\NormalTok{(}\FloatTok{0.124}\NormalTok{, post.pes}\SpecialCharTok{$}\NormalTok{alpha.post, post.pes}\SpecialCharTok{$}\NormalTok{beta.post, }\AttributeTok{lower.tail =} \ConstantTok{FALSE}\NormalTok{)}
\NormalTok{    simdat }\OtherTok{=}\NormalTok{ simdat }\SpecialCharTok{\%\textgreater{}\%}
      \FunctionTok{mutate}\NormalTok{(}\AttributeTok{success.prim =} \FunctionTok{ifelse}\NormalTok{(simdat}\SpecialCharTok{$}\NormalTok{criterion.prim }\SpecialCharTok{\textgreater{}} \FloatTok{0.8}\NormalTok{, }\DecValTok{1}\NormalTok{, }\DecValTok{0}\NormalTok{),}
             \AttributeTok{success.opt =} \FunctionTok{ifelse}\NormalTok{(simdat}\SpecialCharTok{$}\NormalTok{criterion.opt }\SpecialCharTok{\textgreater{}} \FloatTok{0.8}\NormalTok{, }\DecValTok{1}\NormalTok{, }\DecValTok{0}\NormalTok{),}
             \AttributeTok{success.pes =} \FunctionTok{ifelse}\NormalTok{(simdat}\SpecialCharTok{$}\NormalTok{criterion.pes }\SpecialCharTok{\textgreater{}} \FloatTok{0.8}\NormalTok{, }\DecValTok{1}\NormalTok{, }\DecValTok{0}\NormalTok{)}
\NormalTok{             )}
\NormalTok{    result[i,}\StringTok{"Primary"}\NormalTok{] }\OtherTok{=} \DecValTok{1}\SpecialCharTok{/}\NormalTok{N }\SpecialCharTok{*} \FunctionTok{sum}\NormalTok{(simdat}\SpecialCharTok{$}\NormalTok{success.prim)}
\NormalTok{    result[i,}\StringTok{"Optimisitc"}\NormalTok{] }\OtherTok{=} \DecValTok{1}\SpecialCharTok{/}\NormalTok{N }\SpecialCharTok{*} \FunctionTok{sum}\NormalTok{(simdat}\SpecialCharTok{$}\NormalTok{success.opt)}
\NormalTok{    result[i,}\StringTok{"Pessimistic"}\NormalTok{] }\OtherTok{=} \DecValTok{1}\SpecialCharTok{/}\NormalTok{N }\SpecialCharTok{*} \FunctionTok{sum}\NormalTok{(simdat}\SpecialCharTok{$}\NormalTok{success.pes)}
     
    \CommentTok{\#print(simdat)}
\NormalTok{    i }\OtherTok{=}\NormalTok{ i }\SpecialCharTok{+} \DecValTok{1}
\NormalTok{  \}}
  
  \FunctionTok{return}\NormalTok{(result)}
\NormalTok{\}}
\end{Highlighting}
\end{Shaded}

\hypertarget{comparison-of-program-output-to-the-results-in-the-study-protocol}{%
\subsection{Comparison of Program Output to the Results in the Study
Protocol}\label{comparison-of-program-output-to-the-results-in-the-study-protocol}}

Create two tables--one for Type I error and one for Power--that compare
your results with those shown in the protocol so you can verify that
your program produces the correct results.

\begin{Shaded}
\begin{Highlighting}[]
\FunctionTok{qn.sim}\NormalTok{(}\AttributeTok{N=}\DecValTok{10000}\NormalTok{, }\AttributeTok{SampleSize =} \DecValTok{50}\NormalTok{, }\AttributeTok{Theta.test =} \FunctionTok{c}\NormalTok{(}\FloatTok{0.375}\NormalTok{, }\FloatTok{0.1875}\NormalTok{)) }\SpecialCharTok{\%\textgreater{}\%} \FunctionTok{flextable}\NormalTok{()}
\end{Highlighting}
\end{Shaded}

\begin{verbatim}
## Warning: fonts used in `flextable` are ignored because the `pdflatex` engine is
## used and not `xelatex` or `lualatex`. You can avoid this warning by using the
## `set_flextable_defaults(fonts_ignore=TRUE)` command or use a compatible engine
## by defining `latex_engine: xelatex` in the YAML header of the R Markdown
## document.
\end{verbatim}

\global\setlength{\Oldarrayrulewidth}{\arrayrulewidth}

\global\setlength{\Oldtabcolsep}{\tabcolsep}

\setlength{\tabcolsep}{2pt}

\renewcommand*{\arraystretch}{1.5}



\providecommand{\ascline}[3]{\noalign{\global\arrayrulewidth #1}\arrayrulecolor[HTML]{#2}\cline{#3}}

\begin{longtable}[c]{|p{0.75in}|p{0.75in}|p{0.75in}|p{0.75in}}



\ascline{1.5pt}{666666}{1-4}

\multicolumn{1}{>{\raggedleft}m{\dimexpr 0.75in+0\tabcolsep}}{\textcolor[HTML]{000000}{\fontsize{11}{11}\selectfont{theta}}} & \multicolumn{1}{>{\raggedleft}m{\dimexpr 0.75in+0\tabcolsep}}{\textcolor[HTML]{000000}{\fontsize{11}{11}\selectfont{Primary}}} & \multicolumn{1}{>{\raggedleft}m{\dimexpr 0.75in+0\tabcolsep}}{\textcolor[HTML]{000000}{\fontsize{11}{11}\selectfont{Optimisitc}}} & \multicolumn{1}{>{\raggedleft}m{\dimexpr 0.75in+0\tabcolsep}}{\textcolor[HTML]{000000}{\fontsize{11}{11}\selectfont{Pessimistic}}} \\

\ascline{1.5pt}{666666}{1-4}\endfirsthead 

\ascline{1.5pt}{666666}{1-4}

\multicolumn{1}{>{\raggedleft}m{\dimexpr 0.75in+0\tabcolsep}}{\textcolor[HTML]{000000}{\fontsize{11}{11}\selectfont{theta}}} & \multicolumn{1}{>{\raggedleft}m{\dimexpr 0.75in+0\tabcolsep}}{\textcolor[HTML]{000000}{\fontsize{11}{11}\selectfont{Primary}}} & \multicolumn{1}{>{\raggedleft}m{\dimexpr 0.75in+0\tabcolsep}}{\textcolor[HTML]{000000}{\fontsize{11}{11}\selectfont{Optimisitc}}} & \multicolumn{1}{>{\raggedleft}m{\dimexpr 0.75in+0\tabcolsep}}{\textcolor[HTML]{000000}{\fontsize{11}{11}\selectfont{Pessimistic}}} \\

\ascline{1.5pt}{666666}{1-4}\endhead



\multicolumn{1}{>{\raggedleft}m{\dimexpr 0.75in+0\tabcolsep}}{\textcolor[HTML]{000000}{\fontsize{11}{11}\selectfont{0.3750}}} & \multicolumn{1}{>{\raggedleft}m{\dimexpr 0.75in+0\tabcolsep}}{\textcolor[HTML]{000000}{\fontsize{11}{11}\selectfont{0.9996}}} & \multicolumn{1}{>{\raggedleft}m{\dimexpr 0.75in+0\tabcolsep}}{\textcolor[HTML]{000000}{\fontsize{11}{11}\selectfont{1.0000}}} & \multicolumn{1}{>{\raggedleft}m{\dimexpr 0.75in+0\tabcolsep}}{\textcolor[HTML]{000000}{\fontsize{11}{11}\selectfont{0.9993}}} \\





\multicolumn{1}{>{\raggedleft}m{\dimexpr 0.75in+0\tabcolsep}}{\textcolor[HTML]{000000}{\fontsize{11}{11}\selectfont{0.1875}}} & \multicolumn{1}{>{\raggedleft}m{\dimexpr 0.75in+0\tabcolsep}}{\textcolor[HTML]{000000}{\fontsize{11}{11}\selectfont{0.7455}}} & \multicolumn{1}{>{\raggedleft}m{\dimexpr 0.75in+0\tabcolsep}}{\textcolor[HTML]{000000}{\fontsize{11}{11}\selectfont{0.9679}}} & \multicolumn{1}{>{\raggedleft}m{\dimexpr 0.75in+0\tabcolsep}}{\textcolor[HTML]{000000}{\fontsize{11}{11}\selectfont{0.6097}}} \\

\ascline{1.5pt}{666666}{1-4}



\end{longtable}



\arrayrulecolor[HTML]{000000}

\global\setlength{\arrayrulewidth}{\Oldarrayrulewidth}

\global\setlength{\tabcolsep}{\Oldtabcolsep}

\renewcommand*{\arraystretch}{1}

\begin{Shaded}
\begin{Highlighting}[]
\FunctionTok{qn.sim}\NormalTok{(}\AttributeTok{N=}\DecValTok{10000}\NormalTok{, }\AttributeTok{SampleSize =} \DecValTok{50}\NormalTok{, }\AttributeTok{Theta.test =} \FunctionTok{c}\NormalTok{(}\FloatTok{0.025}\NormalTok{, }\FloatTok{0.05}\NormalTok{, }\FloatTok{0.075}\NormalTok{, }\FloatTok{0.10}\NormalTok{, }\FloatTok{0.124}\NormalTok{)) }\SpecialCharTok{\%\textgreater{}\%} \FunctionTok{flextable}\NormalTok{()}
\end{Highlighting}
\end{Shaded}

\begin{verbatim}
## Warning: fonts used in `flextable` are ignored because the `pdflatex` engine is
## used and not `xelatex` or `lualatex`. You can avoid this warning by using the
## `set_flextable_defaults(fonts_ignore=TRUE)` command or use a compatible engine
## by defining `latex_engine: xelatex` in the YAML header of the R Markdown
## document.
\end{verbatim}

\global\setlength{\Oldarrayrulewidth}{\arrayrulewidth}

\global\setlength{\Oldtabcolsep}{\tabcolsep}

\setlength{\tabcolsep}{2pt}

\renewcommand*{\arraystretch}{1.5}



\providecommand{\ascline}[3]{\noalign{\global\arrayrulewidth #1}\arrayrulecolor[HTML]{#2}\cline{#3}}

\begin{longtable}[c]{|p{0.75in}|p{0.75in}|p{0.75in}|p{0.75in}}



\ascline{1.5pt}{666666}{1-4}

\multicolumn{1}{>{\raggedleft}m{\dimexpr 0.75in+0\tabcolsep}}{\textcolor[HTML]{000000}{\fontsize{11}{11}\selectfont{theta}}} & \multicolumn{1}{>{\raggedleft}m{\dimexpr 0.75in+0\tabcolsep}}{\textcolor[HTML]{000000}{\fontsize{11}{11}\selectfont{Primary}}} & \multicolumn{1}{>{\raggedleft}m{\dimexpr 0.75in+0\tabcolsep}}{\textcolor[HTML]{000000}{\fontsize{11}{11}\selectfont{Optimisitc}}} & \multicolumn{1}{>{\raggedleft}m{\dimexpr 0.75in+0\tabcolsep}}{\textcolor[HTML]{000000}{\fontsize{11}{11}\selectfont{Pessimistic}}} \\

\ascline{1.5pt}{666666}{1-4}\endfirsthead 

\ascline{1.5pt}{666666}{1-4}

\multicolumn{1}{>{\raggedleft}m{\dimexpr 0.75in+0\tabcolsep}}{\textcolor[HTML]{000000}{\fontsize{11}{11}\selectfont{theta}}} & \multicolumn{1}{>{\raggedleft}m{\dimexpr 0.75in+0\tabcolsep}}{\textcolor[HTML]{000000}{\fontsize{11}{11}\selectfont{Primary}}} & \multicolumn{1}{>{\raggedleft}m{\dimexpr 0.75in+0\tabcolsep}}{\textcolor[HTML]{000000}{\fontsize{11}{11}\selectfont{Optimisitc}}} & \multicolumn{1}{>{\raggedleft}m{\dimexpr 0.75in+0\tabcolsep}}{\textcolor[HTML]{000000}{\fontsize{11}{11}\selectfont{Pessimistic}}} \\

\ascline{1.5pt}{666666}{1-4}\endhead



\multicolumn{1}{>{\raggedleft}m{\dimexpr 0.75in+0\tabcolsep}}{\textcolor[HTML]{000000}{\fontsize{11}{11}\selectfont{0.025}}} & \multicolumn{1}{>{\raggedleft}m{\dimexpr 0.75in+0\tabcolsep}}{\textcolor[HTML]{000000}{\fontsize{11}{11}\selectfont{0.0000}}} & \multicolumn{1}{>{\raggedleft}m{\dimexpr 0.75in+0\tabcolsep}}{\textcolor[HTML]{000000}{\fontsize{11}{11}\selectfont{0.0089}}} & \multicolumn{1}{>{\raggedleft}m{\dimexpr 0.75in+0\tabcolsep}}{\textcolor[HTML]{000000}{\fontsize{11}{11}\selectfont{0.0000}}} \\





\multicolumn{1}{>{\raggedleft}m{\dimexpr 0.75in+0\tabcolsep}}{\textcolor[HTML]{000000}{\fontsize{11}{11}\selectfont{0.050}}} & \multicolumn{1}{>{\raggedleft}m{\dimexpr 0.75in+0\tabcolsep}}{\textcolor[HTML]{000000}{\fontsize{11}{11}\selectfont{0.0031}}} & \multicolumn{1}{>{\raggedleft}m{\dimexpr 0.75in+0\tabcolsep}}{\textcolor[HTML]{000000}{\fontsize{11}{11}\selectfont{0.1058}}} & \multicolumn{1}{>{\raggedleft}m{\dimexpr 0.75in+0\tabcolsep}}{\textcolor[HTML]{000000}{\fontsize{11}{11}\selectfont{0.0003}}} \\





\multicolumn{1}{>{\raggedleft}m{\dimexpr 0.75in+0\tabcolsep}}{\textcolor[HTML]{000000}{\fontsize{11}{11}\selectfont{0.075}}} & \multicolumn{1}{>{\raggedleft}m{\dimexpr 0.75in+0\tabcolsep}}{\textcolor[HTML]{000000}{\fontsize{11}{11}\selectfont{0.0312}}} & \multicolumn{1}{>{\raggedleft}m{\dimexpr 0.75in+0\tabcolsep}}{\textcolor[HTML]{000000}{\fontsize{11}{11}\selectfont{0.3274}}} & \multicolumn{1}{>{\raggedleft}m{\dimexpr 0.75in+0\tabcolsep}}{\textcolor[HTML]{000000}{\fontsize{11}{11}\selectfont{0.0108}}} \\





\multicolumn{1}{>{\raggedleft}m{\dimexpr 0.75in+0\tabcolsep}}{\textcolor[HTML]{000000}{\fontsize{11}{11}\selectfont{0.100}}} & \multicolumn{1}{>{\raggedleft}m{\dimexpr 0.75in+0\tabcolsep}}{\textcolor[HTML]{000000}{\fontsize{11}{11}\selectfont{0.1218}}} & \multicolumn{1}{>{\raggedleft}m{\dimexpr 0.75in+0\tabcolsep}}{\textcolor[HTML]{000000}{\fontsize{11}{11}\selectfont{0.5690}}} & \multicolumn{1}{>{\raggedleft}m{\dimexpr 0.75in+0\tabcolsep}}{\textcolor[HTML]{000000}{\fontsize{11}{11}\selectfont{0.0556}}} \\





\multicolumn{1}{>{\raggedleft}m{\dimexpr 0.75in+0\tabcolsep}}{\textcolor[HTML]{000000}{\fontsize{11}{11}\selectfont{0.124}}} & \multicolumn{1}{>{\raggedleft}m{\dimexpr 0.75in+0\tabcolsep}}{\textcolor[HTML]{000000}{\fontsize{11}{11}\selectfont{0.2791}}} & \multicolumn{1}{>{\raggedleft}m{\dimexpr 0.75in+0\tabcolsep}}{\textcolor[HTML]{000000}{\fontsize{11}{11}\selectfont{0.7644}}} & \multicolumn{1}{>{\raggedleft}m{\dimexpr 0.75in+0\tabcolsep}}{\textcolor[HTML]{000000}{\fontsize{11}{11}\selectfont{0.1626}}} \\

\ascline{1.5pt}{666666}{1-4}



\end{longtable}



\arrayrulecolor[HTML]{000000}

\global\setlength{\arrayrulewidth}{\Oldarrayrulewidth}

\global\setlength{\tabcolsep}{\Oldtabcolsep}

\renewcommand*{\arraystretch}{1}

\end{document}
